\chapter{Fn: introducción}
\label{chap:fn-introduccion}

En este capítulo analizaré en profundidad las necesidades y casos de uso que voy a implementar en mi sistema. Servirá como guía del diseño a seguir y da una visión general de las características que es capaz de gestionar la aplicación.

\section{Análisis de requisitos}
\subsection{Requisitos funcionales}

\begin{enumerate}
    \item Administración de funciones
        \begin{enumerate}
            \item Listar funciones que existen en el sistema por línea de comandos.
            \item Listar funciones que existen en el sistema por el panel de control web.
            \item Desplegar una nueva función usando la herramienta.
            \item Ejecutar una llamada a la función desde la interfaz web.
            \item Eliminar funciones por línea de comandos.
        \end{enumerate}
    \item Monitorización de funciones
        \begin{enumerate}
            \item Acceder al número de instancias abiertas de cada función a lo largo del tiempo.
        \end{enumerate}
\end{enumerate}

\subsection{Requisitos no funcionales}
\begin{enumerate}
    \item Autoescalar el número de instancias que sirven una función cuando la carga en espera en la cola supere el tiempo de iniciar una nueva.
    \item Cerrar instancias a medida que la carga se reduzca y sobre capacidad de procesamiento suficiente.
    \item Recibir un contexto preparado para cada lenguaje que soportemos con datos y métodos útiles al ejecutar una función.
    \item Al eliminar una función del sistema deben cerrarse inmediatamente todas las instancias que tengamos en ejecución.
\end{enumerate}

\section{Casos de uso}

\subsection{Panel de control web}

\begin{figure}[H]
    \centering
    \includegraphics[width=\textwidth]{../images/casos-uso/web.png}
    \caption{Casos de uso para el panel de control web}
    \label{fig:cu-web}
\end{figure}

\subsection{Herramienta de línea de comandos \emph{fnctl}}

\begin{figure}[H]
    \centering
    \includegraphics[width=\textwidth]{../images/casos-uso/fnctl.png}
    \caption{Casos de uso para la herramienta de línea de comandos \emph{fnctl}}
    \label{fig:cu-fnctl}
\end{figure}